\input epsf
\centerline{\bf Table of Squares Example}

Here, we are going to take a look at a specific example of how to use
the ENIAC simulator to compute a table of squares and to punch the
output onto cards.
We're going to follow the typical development cycle of first analyzing
the problem to create an algorithm to solve it.
Then we'll create an ENIAC configuration that implemented the
algorithm, and finally translate that configuration into a input
file for the simulator.
The resulting simulator configuration can be found in the {\tt programs}
directory in the file named {\tt sq3.e}.

\beginsection
Problem Analysis

Since the task is to compute a table of $f(x)=x^2$ values for integer values
of $x$, the first thought might be to use the multiplier to multiply
each $x$ value by itself.
However, there's a faster way of doing this by taking advantage of
the familiar equation: $(x+1)^2=x^2+2x+1$.
In other words, if we have the values of $f(x)=x^2$ and $x$, then
all we have to do is add $2x + 1$ to $f(x)$ to get $f(x+1)$.
So in algorithmic terms, we might describe the process as follows:
\medskip
{\obeylines
Initialize the values $f(x)=0$ and $x=0$
For each row of the table:
\quad Add $2x$ to $f(x)$
\quad Add 1 to $f(x)$
\quad Add 1 to $x$
\quad Punch a card with the values of $x$ and $x^2$

}

\beginsection
Implementing on the ENIAC

Because after clearing the machine, all the accumulators are set to~0,
we'll consider our initialization step solved.
For the~1 that shows up in a couple of steps of the algorithm, we'll
set up the constant transmitter with the value~1 and use it for those
cases where we need to add 1.
The next decision we need to make is how the programming of the
master programmer is going to relate to the ``for each'' loop.
I.e.\ is this going to be a pre-test or a post-test loop.
For this example, we're going to choose pre-test.
With these decisions made, we'll configure the ENIAC according
to this figure, explained in notes that follow:
\medskip
\centerline{\epsfxsize=6.5in \epsfbox{sqex.1}}
\medskip
{\advance\leftskip 20pt
\item{1.} The data trunks are the lines above the accumulators and
constant transmitter and are numbered from bottom to top starting
at~1.
\item{2.} The program trunks are below all the units and are numbered
from 1-1 to 1-4 top to bottom.
\item{3.} Since we're treating this as a pre-test loop, the initiating output
of the initiating unit is carried through program trunk 1-1 to the stepper
input (Ci) of the third (C) stepper of the master programmer.
Because of the switch settings we've made, the first 9999 such input
pulses will result in output pulses on the C1o output.
\item{4.} The C1o output goes to program trunk 1-4 that starts the sequence
of operations in the body of the loop.
\item{5.} The first step is to add $2x$ to $x^2$, were $x$ is in Accumulator~13 and
$x^2$ is in Accumulator~20.
Program trunk 1-4 then starts program~6 on both Accumulators~13 and~20.
Accumulator~13 responds by transmitting its value on the A output (to
data trunk 2) twice while
Accumulator~20 receives from data trunk~2 into its $\beta$ input twice.
As a result, we now have $x^2+2x$ in Accumulator~20.
\item{6.} The completion of this operation is signaled on Accumulator~20's~6o
line which is connected to program trunk 1-3.
\item{7.} The signal on program trunk 1-3 initiates program~5 on both
Accumulators~13 and~20 as well as initiating program~26 on the constant
transmitter.
The behavior of these programs is to put the number~1 onto data trunk~1
and to read the value on data trunk~1 into both accumulators via their $\alpha$
inputs.
As a result, we now have $x+1$ in Accumulator~13 and $x^2+2x+1=(x+1)^2$ in
Accumulator~20.
\item{8.} With the completion of these calculations, we can now puch an output
card with the values of $x$ and $x^2$ on it.
The completion of the adds of~1 results in a pulse on the~5o line of Accumulator~20
which is carried to the Pi input of the initiating unit via program trunk 1-2.
\item{9.} The Pi input on the initiating unit starts a card punch cycle.
If the card punch is idle, then the data on the accumultors is read into the
punch interface and a pulse is sent out the Po terminal to indicate that the
next computation can begin.
If the card punch is still busy with the last card, the Po pulse is delayed until
the card punch has completed enough of its operation to allow the accumulator
values to again be copied into the punch interface.
\item{10.} The Po output from the initiating unit is connected to program trunk 1-1
in the same way as the initiating output.
As a result, when the punch control indicates it's time to start a new computation,
the signal on program trunk 1-1 starts the whole process over, just like the
original initiation signal did.

}

\beginsection
Simulator Configuration

The first few lines of our configuration file describe the connections
on data trunk~1 from the constant transmitter output to the $\alpha$
inputs of the two accumulators.
Note that in the file itself, you can specify the input terminal name
with the UTF-8 codepoint for $\alpha$, the lower-case letter `a'
or the word `alpha.'

{\obeylines\tt
p c.o 1
p 1 a13.$\alpha$
p 1 a20.$\alpha$

}

\noindent
The next two lines are the connection from the A output of Accumulator~13
to the $\beta$ input of Accumulator~20.

{\obeylines\tt
p a13.A 2
p 2 a20.$\beta$

}

\noindent
Now we move on to the program control connections.
The first two connect the initiating pulse output to the stepper input
on the master programmer.

{\obeylines\tt
p i.io 1-1
p 1-1 p.Ci

}

\noindent
The completion of the constant transmitter program triggers
the rest of the computations.
This is carried on program trunk 1-4.
As described above, the control then sequences through Program~6
of Accumulators~13 and~20, to Program~5 of both accumulators
to the punch.

{\obeylines\tt
p p.C1o 1-4
p 1-4 a13.6i
p 1-4 a20.6i
p a20.5o 1-2
p a20.6o 1-3
p 1-3 a13.5i
p 1-3 a20.5i
p 1-3 c.26i
p 1-2 i.pi
p i.po 1-1

}

\noindent
Now we set up the switches, the first set of which are the
operation and repeat switches for Programs~5 and~6 on
Accumulators~13 and~20.

{\obeylines\tt
s a13.op5 $\alpha$
s a13.op6 A
s a13.rp5 1
s a13.rp6 2
s a20.op5 $\alpha$
s a20.op6 $\beta$
s a20.rp5 1
s a20.rp6 2

}

\noindent
The last switch settings for the constant transmitter are not
shown on the diagram given here.
They are set so that Program~26 transmits the value of the J
constant, and so that the J constant is set to the value~1.

{\obeylines\tt
s c.s26 Jlr
s c.j1 1

}

\noindent
The next several switch settings are those for the master
programmer as shown in the diagram.
{\obeylines\tt
s p.d17s1 9
s p.d16s1 9
s p.d15s1 9
s p.d14s1 9
s p.d14s2 1
s p.cC 4

}

\noindent
Finally, we have the switch settings on the punch interface
that enable the printing of the values of Accumulators~13 and~20.
{\obeylines\tt
s pr.2 P
s pr.3 P
s pr.15 P
s pr.16 P

}

\vfill\eject\end
